\documentclass[12pt]{article}

\usepackage{fullpage}
\usepackage{multicol,multirow}
\usepackage{tabularx}
\usepackage{ulem}
\usepackage[utf8]{inputenc}
\usepackage{mathtools}
\usepackage{pgfplots}
\usepackage{listings}
\usepackage[russian]{babel}
\usepackage[colorlinks=true, urlcolor=black, linkcolor=red]{hyperref}

\begin{document}

\thispagestyle{empty}
\begin{center}
	МОСКОВСКИЙ АВИАЦИОННЫЙ ИНСТИТУТ

	(НАЦИОНАЛЬНЫЙ ИССЛЕДОВАТЕЛЬСКИЙ УНИВЕРСИТЕТ)
\vspace{3ex}

	Институт №8 «Компьютерные науки и прикладная математика»

	Кафедра №806 «Вычислительна математика и программирование»

\end{center}
\vspace{25ex}
\begin{center}
	\textbf{\large{Курсовая работа по курсу\linebreak \textquotedblleft Дискретный анализ\textquotedblright}}
\end{center}
\vspace{3ex}
\begin{center}
    \textbf{\large{Персистентные структуры данных}}
\end{center}
\vspace{20ex}
\begin{flushright}
	\textit{Студент: } Караев Тариел Жоомартбекович

	\vspace{2ex}
	\textit{Группа: } М8О-308Б-22

	\vspace{2ex}
	\textit{Преподаватель: } Макаров Никита Константинович

	\vspace{2ex}
	\textit{Оценка: } \underline{\quad\quad\quad\quad\quad\quad}

	 \vspace{2ex}
	\textit{Дата: } \underline{\quad\quad\quad\quad\quad\quad}

	\vspace{2ex}
	\textit{Подпись: } \underline{\quad\quad\quad\quad\quad\quad}

\end{flushright}
\vspace{5ex}

\begin{vfill}
	\begin{center}
		Москва, 2024
	\end{center}
\end{vfill}
\newpage

\begin{center}
\section*{Содержание}
\end{center}
\vspace{5ex}
\begin{enumerate}
  \item Репозиторий
  \item Постановка задачи
  \item Метод решения
  \item Описание алгоритма
  \item Описание программы
  \item Тест производительности
  \item Выводы
\end{enumerate}
\newpage


\section*{Репозиторий}
\vspace{2ex}
\url{https://github.com/SempaiTakoo/DA_labs}

\section*{Постановка задачи}

Программа должна читать входные данные из стандартного потока ввода и выводить ответ на стандартный поток вывода. Реализуйте алгоритм $A\star$ для неориентированного графа.
Расстояние между соседями вычисляется как простое евклидово расстояние на плоскости.

\subsection*{Формат ввода}
В первой строке вам даны два числа $n$ и $m$ ($1 \leq n \leq 10^4, 1 \leq m \leq 10^5$) вершин и рёбер в графе. В следующих $n$ строках вам даны пары чисел $х$, $у$ ($-10^9 \leq x, y \leq 10^9$), описывающие положение вершин графа в двумерном пространстве. В следующих $m$ строках даны пары чисел в отрезке от $1$ до $n$, описывающие рёбра графа. Далее дано число $q$ ($1 \leq q \leq 300$) и в следующих $q$ строках даны запросы в виде пар чисел $a$, $b$ ($1 \leq а, b \leq n$) на поиск кратчайшего пути между двумя вершинами.

\subsection*{Формат вывода}
В ответ на каждый запрос выведите единственное число — длину кратчайшего пути между заданными вершинами с абсолютной либо относительной точностью $10^{-6}$, если пути между вершинами не существует выведите $-1$.

\newpage
\section*{Метод решения}
Как сказано в условии задачи, необходимо реализовать алгоритм $A\star$, позволяющий находить кратчайшие пути между вершинами графа. В качестве эвристической функции используется евклидово расстояние на плоскости.

В процессе работы алгоритм поддерживает две структуры: множество "открытых" вершин, которые ещё предстоит обработать, и множество "закрытых" вершин, которые уже обработаны. Пока множество "открытых" вершин непустое, из него извлекается вершина с наименьшим значением оценки $f$, после чего анализируются её соседи. Если новая оценка пути до соседней вершины оказывается меньше ранее известной, оценка пути обновляется и вершина добавляется во множество "открытых" вершин.

Применение эвристики позволяет алгоритму $A\star$ быстрее находить оптимальный маршрут, избегая лишних обходов, что положительно выделяет его на фоне алгоритма Дийкстры, чьим обобщением он является.

Формально, алгоритм $A\star$ имеет сложность $O((n + m)\log{n}$, так как при использовании очереди с приоритетом требуется $n + m$ раз вставить и извлечь вершину графа со сложностью $\log{n}$. Однако благодаря эвристике реальная скорость $A\star$ близится к $O(n)$.

\newpage
\section*{Описание алгоритма}
Основные этапы работы алгоритма:

\begin{itemize}
    \item Инициализация:
    \begin{enumerate}
        \item Создаются множества "открытых" и "закрытых" вершин.
        \item Для начальной вершины устанавливается нулевая стоимость пути, а её эвристическая оценка рассчитывается на основе евклидова расстояния до конечной вершины.
    \end{enumerate}

    \item Основной цикл поиска:
    \begin{enumerate}
        \item Из "открытого" множества выбирается вершина с наименьшим значением функции оценки.
        \item Если это целевая вершина, алгоритм завершается, возвращая найденную стоимость пути.
        \item Вершина переносится в "закрытое" множество, после чего анализируются все её соседние вершины.
    \end{enumerate}

    \item Обновление данных соседних вершин:
    \begin{enumerate}
        \item Если сосед уже находится в "закрытом" множестве, он пропускается.
        \item Для каждой соседней вершины пересчитывается стоимость пути, и если найден более короткий маршрут, обновляются соответствующие данные.
        \item Соседняя вершина добавляется в "открытое" множество и её приоритет обновляется.
    \end{enumerate}
\end{itemize}

В случае отсутствия пути между вершинами алгоритм $A\star$ корректно возвращает отрицательное значение, что означает недостижимость цели.

\newpage
\section*{Описание программы}

\begin{itemize}
    \item \textbf{Graph} – класс, представляющий граф. Содержит список вершин с координатами и список смежности для рёбер.
    \begin{itemize}
        \item \textbf{addVertex} – метод, добавляющий вершину с заданными координатами.
        \item \textbf{addEdge} – метод, создающий ребро между двумя вершинами, дополнительно вычисляя расстояние между ними.
    \end{itemize}

    \item \textbf{Node} – вспомогательная структура для приоритетной очереди в алгоритме $A\star$, содержащая номер вершины и оценку стоимости пути $f$.

    \item \textbf{get\_distance} – вычисляет евклидово расстояние между двумя точками.

    \item \textbf{main} – точка входа в программу, отвечает за инициализацию и запуск основного процесса вычислений.
    \item \textbf{input\_graph} – функция для считывания данных о графе из входного потока.
    \item \textbf{shortest\_paths\_between\_vertices} – управляет процессом поиска кратчайших путей между заданными парами вершин, используя алгоритм $A\star$.
    \item \textbf{a\_star} – основная функция реализации алгоритма $A\star$, осуществляет поиск кратчайшего пути в графе.
\end{itemize}

\subsection*{Исходныйкод}
\begin{lstlisting}[language=C++, basicstyle=\ttfamily\footnotesize, breaklines=true]
#include <iostream>
#include <vector>
#include <queue>
#include <map>
#include <limits>
#include <set>
#include <cmath>
#include <set>
#include <iomanip>

using namespace std;

const bool DEBUG = false;
const double INF = numeric_limits<double>::infinity();

double get_distance(pair<int, int> a, pair<int, int> b) {
    long x = b.first - a.first;
    long y = b.second - a.second;
    return sqrt(x * x + y * y);
}

class Graph {
public:
    vector<pair<double, double>> vertices;
    vector<vector<pair<int, double>>> adjacency_list;

    Graph(int vertices_count) {
        vertices.resize(vertices_count);
        adjacency_list.resize(vertices_count);
    }

    void addVertex(int vertex_number, double x, double y) {
        vertices[vertex_number] = {x, y};
    }

    void addEdge(int s, int f) {
        double distance = get_distance(vertices[s], vertices[f]);
        adjacency_list[s].push_back({f, distance});
        adjacency_list[f].push_back({s, distance});
    }
};

struct Node {
    int vertex_number;
    double f;

    bool operator>(const Node node) const {
        return f > node.f;
    }
};

Graph input_graph() {
    int n, m;
    cin >> n >> m;

    Graph graph(n);

    int x, y;
    for (int i = 0; i < n; ++i)
    {
        cin >> x >> y;
        graph.addVertex(i, x, y);
    }

    int s, f;
    for (int i = 0; i < m; ++i)
    {
        cin >> s >> f;
        graph.addEdge(s - 1, f - 1);
    }

    return graph;
}

double a_star(Graph& graph, int start, int finish) {
    size_t dim = graph.vertices.size();

    priority_queue<Node, vector<Node>, greater<Node>> open;
    vector<bool> closed(dim, false);
    vector<double> g(dim, INF);
    vector<double> f(dim, INF);

    g[start] = 0;
    f[start] = get_distance(graph.vertices[start], graph.vertices[finish]);
    open.push({start, f[start]});

    while (!open.empty()) {
        Node node = open.top();
        int current = node.vertex_number;
        open.pop();
        closed[current] = true;

        if (current == finish) {
            return g[current];
        }

        for (pair<int, double> edge : graph.adjacency_list[current]) {
            int neighbor = edge.first;
            double distance = edge.second;

            if (closed[neighbor]) {
                continue;
            }

            double g_temp = g[current] + distance;
            if (g_temp >= g[neighbor]) {
                continue;
            }

            g[neighbor] = g_temp;
            f[neighbor] = g[neighbor] + get_distance(
                graph.vertices[neighbor],
                graph.vertices[finish]
            );
            open.push({neighbor, f[neighbor]});
        }
    }

    return -1;
}

void shortest_paths_between_vertices() {
    Graph graph = input_graph();

    int q;
    cin >> q;

    for (int i = 0; i < q; ++i) {
        int start, finish;
        cin >> start >> finish;

        --start;
        --finish;

        double answer = a_star(graph, start, finish);

        if (answer == -1) {
            cout << -1 << '\n';
        } else {
            cout << fixed << setprecision(6) << answer << '\n';
        }
    }
}

int main() {
    ios::sync_with_stdio(false);
    cin.tie(0);
    cout.tie(0);

    shortest_paths_between_vertices();

    return 0;
}
\end{lstlisting}

\newpage
\section*{Тест производительности}
В данной главе представлены результаты тестирования производительности алгоритма $A\star$ на случайных графах. Были проведены тесты при зафиксированных $m = 2n$, $q = 1$ и изменении $n = \alpha \cdot 10^4$, $\alpha = 1, 2, ..., 10$. Данные приведены в таблице 1.

\begin{table}[h]
    \centering
    \begin{tabular}{|c|c|c|}
        \hline
        \toprule
        $n$ & $(n + m)\log(n)$ & Время выполнения $10^{-7}$ с \\
        \midrule
        \hline
        10000  & 120000       & 28866,6 \\
        20000  & 258061,7997  & 50367,6 \\
        30000  & 402940,9129  & 54438,6 \\
        40000  & 552247,199   & 26760,4 \\
        50000  & 704845,5007  & 86821,7 \\
        60000  & 860067,2251  & 199223  \\
        70000  & 1017470,588  & 153932  \\
        80000  & 1176741,597  & 80430,6 \\
        90000  & 1337645,478  & 508806  \\
        100000 & 1500000      & 82967,5 \\
        \bottomrule
        \hline
    \end{tabular}
    \caption{Результаты тестирования производительности}
    \label{tab:performance}
\end{table}

Далее на рисунке 1 изображены два графика:
\begin{itemize}
    \item график функции $(n + m)\log(n)$, являющийся асимптотической верхней границей сложности алгоритма $A\star$,
    \item график зависимости времени выполнения программы от размера входных данных.
\end{itemize}

\begin{figure}[h]
    \centering
    \begin{tikzpicture}
        \begin{axis}[
            width=12cm,
            height=8cm,
            xlabel={$n$},
            ylabel={Значение},
            legend pos=north west,
            grid=major,
        ]
        \addplot[color=blue, mark=*] coordinates {
            (10000, 120000) (20000, 258061.7997) (30000, 402940.9129) (40000, 552247.199) (50000, 704845.5007)
            (60000, 860067.2251) (70000, 1017470.588) (80000, 1176741.597) (90000, 1337645.478) (100000, 1500000)
        };
        \addlegendentry{$(n + m)\log(n)$}

        \addplot[color=red, mark=square*] coordinates {
            (10000, 28866.6) (20000, 50367.6) (30000, 54438.6) (40000, 26760.4) (50000, 86821.7)
            (60000, 199223) (70000, 153932) (80000, 80430.6) (90000, 508806) (100000, 82967.5)
        };
        \addlegendentry{Время выполнения $10^{-7}$ с}

        \end{axis}
    \end{tikzpicture}
    \caption{График зависимости $(n+m)\log(n)$ и времени выполнения от $n$}
    \label{fig:performance_plot}
\end{figure}

Как видно по графикам, благодаря представлению вершин графа в виде координат и эвристике, выражающейся в евклидовом расстоянии на плоскости, алгоритм работает гораздо быстрее своей асимптотической верхней границы $(n+m)\log(n)$. Это обусловлено тем, что эвристика позволяет делать меньше лишних обходов вершин.

\newpage
\section*{Выводы}
В ходе выполнения данной курсовой работы был реализован алгоритм $A\star$ для поиска кратчайшего пути между вершинами графа. Было изучено, что в определённых задачах дополнительная информация в виде эвристики позволяет существенно улучшить эффективность алгоритма.

Также в ходе работы были получены навыки оптимизации программ на C++. В частности, более вдумчивый подход при выборе различных типов и структур данных, которые могут значительно снизить издержки по памяти и ускорить работу программы.

\end{document}
