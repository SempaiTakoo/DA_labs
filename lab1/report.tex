\documentclass[12pt]{article}

\usepackage{listings}
\usepackage{fullpage}
\usepackage{multicol,multirow}
\usepackage{tabularx}
\usepackage{ulem}
\usepackage[utf8]{inputenc}
\usepackage[russian]{babel}

\begin{document}

\section*{Лабораторная работа №\,1 по курсу дискрeтного анализа: сортировка за линейное время}

Выполнил студент группы М80-208Б-22 МАИ \textit{Караев Тариель}.

\subsection*{Условие}

Кратко описывается задача:
\begin{enumerate}
\item
    Требуется разработать программу, осуществляющую ввод пар "ключ-значение", сортировку по возрастанию ключа указанным алгоритмом сортировки за линейное время и вывод получившейся последовательности.
\item
    Вариант задания:
        \begin{itemize}
            \item Сортировка подсчётом.
            \item Тип ключа: числа от 0 до 65535.
            \item Тип значения: числа от 0 до $2^{64} - 1$.
        \end{itemize}
\end{enumerate}

\subsection*{Метод решения}

Так как в условии не сказано, сколько конкретно пар "ключ-значение" будет подано в программу, необходимо реализовать динамический массив -- он же вектор -- способный менять свой размер для хранения данных.
\par Далее требуется реализовать сортировку подсчётом. Так как по условию сортировка должна быть устойчивой, а мы имеем дело не с простыми числами, а с парами "ключ-значение", необходимо конвертировать счётчик в массив префиксных сумм. Это нужно, чтобы при создании массива-результата мы могли знать, в какой индекс необходимо класть пары "ключ-значение".

\subsection*{Описание программы}

Разделение по файлам, описание основных типов данных и функций. \\
\begin{itemize}
    \item
        Класс TVector. Реализованы конструкторы, основные методы, также перегружены некоторые операторы вроде индексатора.
        \begin{lstlisting}[language=C++]
        template <typename T>
        class TVector {
            T* data;
            size_t size, capacity;

           public:
            TVector();
            TVector(size_t initSize);
            TVector(const TVector& other);
            TVector(TVector&& other);
            ~TVector();

            bool isEmpty();
            void Resize(size_t newSize);
            void PushBack(const T& newElement);
            void popBack();

            T* Data() const;
            size_t Size() const;
            size_t Capacity() const;

            T& operator[](size_t index);
            const T& operator[](size_t index) const;

           private:
            static T* Allocate(size_t newCapacity);
            void Reallocate(size_t newCapacity);
            void Destroy(T *object);
        };
        \end{lstlisting}
    \item
        Структура TPair. Необходим для хранения пар "ключ-значение".
        \begin{lstlisting}[language=C++]
        struct TPair {
            uint16_t key;
            uint64_t value;
        };
        \end{lstlisting}
    \item
        Функция main. Здесь реализован алгоритм сортировки подсчётом. Работа ведётся с счётчиком counter и массивом данных data.
        \begin{lstlisting}[language=C++]
        int main() {
            std::ios_base::sync_with_stdio(false);
            std::cin.tie(0);

            const size_t counterSize = UINT16_MAX + 1;
            TVector<size_t> counter(counterSize);
            for (size_t i = 0; i < counterSize; ++i) {
                counter[i] = 0;
            }

            TVector<TPair> data;
            TPair pair;
            while (std::cin >> pair.key >> pair.value) {
                ++counter[pair.key];
                data.PushBack(pair);
            }

            for (size_t i = 1; i < counter.Size(); ++i) {
                counter[i] = counter[i - 1] + counter[i];
            }

            TVector<TPair> result(data.Size());
            for (size_t i = data.Size(); i-- > 0; ) {
                uint16_t key = data[i].key;
                result[counter[key] - 1] = data[i];
                --counter[key];
            }

            for (size_t i = 0; i < result.Size(); ++i) {
                std::cout << result[i].key << '\t'
                          << result[i].value << '\n';
            }

            return 0;
        }
        \end{lstlisting}
\end{itemize}

\subsection*{Дневник отладки}

Изначально предполагался иной алгоритм, в которым должны были использоваться двумерные векторы и, возможно, очереди. Однако, в виду нежелания заниматься реализацией дополнительных структур данных, было принято решение использовать алгоритм, подразумевающий только векторы.

\subsection*{Тест производительности}

Померить время работы кода лабораторной и теста производительности
на разных объемах входных данных. Сравнить результаты. Проверить,
что рост времени работы при увеличении объема входных данных
согласуется с заявленной сложностью.

Сортировка подсчётом работает за линейное время. Если обратить на код программы, то видно, что функция main состоит из нескольких последовательных циклов. Для большей наглядности приведём таблицу, в которой написанная сортировка сравнивается со стандартными функция языка C++.

\begin{center}
\begin{tabular}{ |c|c|c|c| }
    \hline
    Количество пар "ключ-значение" & Sort(), мс & std::sort(), мс & std::stable$\_$sort(), мс \\
    \hline
    1000 & 1247 & 280 & 187 \\
    10000 & 2167 & 3633 & 2206 \\
    100000 & 6974 & 38296 & 28208 \\
    1000000 & 55033 & 425589 & 341039 \\
    10000000 & 619328 & 5108635 & 3903731 \\
    \hline
    \end{tabular}
\end{center}

На больших объёмах входных данных (от ста тысяч до десяти миллионов) становится заметно, что время сортировки пропорционально количеству пар "ключ-значение". Причём написанная функция оказывается заметно быстрее стандартных функций языка C++, так как те используют алгоритмы с временной сложностью $n \log{n}$.

Ниже приведена программа benchmark.cpp, использовавшаяся для засечения времени работы функций:
\begin{lstlisting}[language=C++]
bool cmp(const std::pair<size_t, size_t>& a,
         const std::pair<size_t, size_t>& b) {
    return a.first < b.first;
}

int main() {
    TVector<TPair> data;
    std::vector<std::pair<size_t, size_t>> benchmarkData;

    TPair pair;
    while (std::cin >> pair.key >> pair.value) {
        data.PushBack(pair);
        benchmarkData.push_back({pair.key, pair.value});
    }

    auto start1 = std::chrono::high_resolution_clock::now();
    Sort(data);
    auto finish1 = std::chrono::high_resolution_clock::now();

    auto start2 = std::chrono::high_resolution_clock::now();
    sort(benchmarkData.begin(), benchmarkData.end(), cmp);
    auto finish2 = std::chrono::high_resolution_clock::now();

    auto start3 = std::chrono::high_resolution_clock::now();
    stable_sort(benchmarkData.begin(), benchmarkData.end(), cmp);
    auto finish3 = std::chrono::high_resolution_clock::now();

    auto duration1 =
        std::chrono::duration_cast<std::chrono::microseconds>(finish1
                                                              - start1);
    auto duration2 =
        std::chrono::duration_cast<std::chrono::microseconds>(finish2
                                                              - start2);
    auto duration3 =
        std::chrono::duration_cast<std::chrono::microseconds>(finish3
                                                              - start3);
    std::cout << "my sort duration: " << duration1.count() << std::endl;
    std::cout << "std sort duration: " << duration2.count() << std::endl;
    std::cout << "std stable_sort duration: " << duration3.count()
              << std::endl;

    return 0;
}

\end{lstlisting}

\subsection*{Выводы}

В ходе выполнения программы был реализован алгоритм сортировки подсчётом. Данная сортировка имеет ограничение, однако даёт преимущество в виде линейной скорости работы. На мой взгляд, в большинстве случаев ограничение не столь значительно. Что самое важное: данная сортировка довольна проста в понимании и реализации.

Помимо этого был получен опыт реализации шаблонных структур данных.

\end{document}
